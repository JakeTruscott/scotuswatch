
\noindent It feels like much longer than a year ago when the Court decided major cases dealing with Affirmative Action and the First Amendment, but here we are, at the end of another term. This time at the beginning of July -- which, even as it concluded on the first day of the month, is a particularly rare occurence. Aside from the COVID terms (OT 2019 and 2020) the last time the Court ended in July was OT 1995 -- a decade before John Roberts assumed the role of Chief Justice.  \\

Many were focused this term on the two election-related cases, both of which included former President Donald Trump as a named party (\emph{Trump v. Anderson} and \emph{Trump v. United States}). While the Justices ultimately maintained a reasonable level of consensus in the former (\emph{Anderson}), clear ideological differences culminated in the latter (\emph{Trump v. US}). A similar outcome emerged with perhaps the term's most notable shift to existing precedent with the overturning of \emph{Chevron} (1984) in \emph{Loper Bright}. \\

In many areas, the Justices appeared to meet the expectations drawn from behaviors in previous terms. These included the backlogged docket with the majority of cases decided in May and June, several ideologically split decisions, and most decisions with split opinions being released in the final weeks of the term. Yet, that's not to say there weren't a collection of suprises. For example, the accidental early release of \emph{Moyle} in late-June also provided a momentary shock to observers as the Court acknowledged the rare clerical error. More substantively, a pattern of early unanimity continued into the end of April at the highest rate in well over a century. However, it is important to recognize that while the concluding rate of unanimity (approximately 46\% of all decisions in the 2023 term) is among the highest in recent memory, the rate (or consistency) of unanimity began to taper as the term neared its conclusion. \\

Since the creation of the Court's current conservative supermajority in 2020, this term saw the most ideological splits of any term aside from OT 2021 (when the Court decided landmark cases like \emph{Dobbs} and \emph{Bruen}). Interestingly, while the notion of a (6-3) decision might preclude an inevitable split along ideological lines, this term again proved that to not always be the case. Unlike OT 2020 and 2021 -- where ideologically split (6-3) decisions were nearly double and triple the rate of non-ideologically split (6-3) decisions, respectively -- this term saw more (6-3) decisions with blurred ideological coalitions. Of the twenty-two (6-3) decisions released this term, half (eleven) maintained the established six conservative versus three liberal split -- a pattern similar to last term's five decisions with ideological splits of the total eleven reached by a (6-3) coalition. \\

Ideologically split decisions tend to be released towards the end of each Supreme Court term because of the number, extent, and length of separate (concurring and dissenting) opinions. This term was no different, with the longest individual opinions of the term released in the final weeks -- including Justice Kavanaugh’s dissent in \emph{Purdue Pharma}, Justice Roberts' majority in \emph{Trump v. U.S.}, Justice Sotomayor’s dissent in \emph{SEC v. Jarkesy}, and several of the opinions in \emph{Loper Bright}. Along similar lines, the decision with the most separate opinions (\emph{Rahimi} with seven) was released in the term’s last full week and included five concurrences along with a dissent from Justice Thomas and Chief Justice Roberts’ majority opinion. \\

In terms of overall opinion authorship, Justices Thomas and Jackson both wrote the most with twenty-one. While each wrote eleven concurrences, the separation emerges with Thomas authoring seven majority opinions to Jackson's five, as well as Thomas authoring five dissents to Jackson's seven. Chief Justice Roberts authored the fewest with nine total. Justices Sotomayor and Jackson wrote the most dissents with seven.  \\

Some of the most important statistics have not shifted much since Justice Barrett joined the Court in 2020. While Justice Kavanaugh was most frequently in the majority last term, this term he was second at (approx.) 95\%, while Chief Justice Roberts -- who was the second most frequent in the majority last term -- was the most frequent this term at 96\%. Justice Barrett was again the third most frequent in the majority at 92\% of the time. The liberal bloc of justices (Kagan, Sotomayor, and Jackson) -- who were not the least frequent in the majority last term -- dipped in their majority frequencies this term and were the three least frequent in the majority with Justices Kagan and Sotomayor tied at the bottom with 71\%. \\

Compared to prior terms, the Justices’ alignments were mostly similar at the high and low end this term with Justices Sotomayor and Kagan agreeing most often at 97\% of the time, followed by Chief Justice Roberts and Justice Kavanaugh at 95\%. At the low end, both Justices Kagan and Sotomayor agreed with Justice Thomas least frequently in only 50\% of all decisions. \\

Even with Justices Barrett and Sotomayor beginning the term with combined speaking engagements discussing the Justices’ shared values notwithstanding differing modes of legal interpretation, the perceivable differences in visions between the Justices were once again put on display by the conclusion of the term.  The tenor of Justice Kagan’s dissent in \emph{Loper Bright} and Justice Jackson’s concurrence and dissent \emph{in part} in \emph{Moyle} present two examples of the divergent approaches existing between the Justices on major issues. These do not appear to be reparable rifts in core beliefs. As long as this remains the case -- and as long as the Justices take on contentious cases affecting broad swaths of the public -- we will likely continue to focus on the differences of opinion between the Justices, rather than on their points of consensus.
